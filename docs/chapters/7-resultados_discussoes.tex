\section{Resultados Obtidos}
Esta seção apresenta os principais resultados obtidos com o desenvolvimento do projeto, bem como os desafios enfrentados durante sua execução. Também são discutidas as contribuições técnicas e as possibilidades de evolução da solução proposta.

\subsection{Resultados}
O desenvolvimento do \textit{Prism} resultou em uma aplicação funcional capaz de realizar recomendações personalizadas de filmes, com autenticação de usuários, armazenamento em nuvem e integração dinâmica com \textit{APIs} externas. O sistema apresenta uma interface moderna e responsiva, permitindo ao usuário cadastrar-se, favoritar títulos e visualizar sugestões personalizadas conforme suas preferências.

A integração com a \textbf{\textit{IMDb API}} possibilitou o acesso a informações atualizadas e detalhadas sobre filmes e séries, enquanto o uso da \textbf{\textit{API Gemini}} trouxe um diferencial na personalização das recomendações, permitindo que o sistema compreendesse solicitações em linguagem natural e sugerisse conteúdos de forma contextualizada.

Durante o desenvolvimento, destacaram-se os ganhos de produtividade obtidos pelo uso do \textbf{\textit{Flutter}}, com recursos como o \textit{Hot Reload}, e pela adoção do \textbf{\textit{Firebase}}, que simplificou a estrutura do \textit{backend} e reduziu custos de infraestrutura. O uso do \textbf{\textit{GitHub}} garantiu versionamento e colaboração eficiente entre os membros da equipe, contribuindo para um processo de desenvolvimento mais ágil e organizado.

\subsection{Discussões}
Os principais desafios encontrados envolveram o gerenciamento da comunicação entre as \textit{APIs} externas e o tratamento das respostas da inteligência artificial, além da definição de critérios para priorização de recomendações. Apesar dessas dificuldades, o projeto demonstrou ser tecnicamente viável e com alto potencial de escalabilidade.

A combinação entre tecnologias \textit{serverless} e inteligência artificial provou-se eficaz para construir um sistema ágil e personalizável, abrindo caminho para futuras evoluções. Entre as possíveis melhorias estão a inclusão de análises de comportamento do usuário, recomendações baseadas em histórico de visualização e integração com plataformas de \textit{streaming} reais. Dessa forma, o \textit{Prism} pode se consolidar como uma ferramenta completa para personalização de conteúdo audiovisual, aliando inovação tecnológica e experiência do usuário.
