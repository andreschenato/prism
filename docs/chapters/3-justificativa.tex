\section{Justificativa}
A consolidação das plataformas de streaming transformou o consumo de conteúdo audiovisual, disponibilizando vastos catálogos de filmes e séries. Contudo, essa abundância gerou um desafio para os usuários: a dificuldade em encontrar conteúdo relevante e personalizado em meio a inúmeras opções, levando à sobrecarga de escolha. Sistemas de recomendação atuais, muitas vezes, não conseguem atender plenamente à demanda por sugestões altamente alinhadas às preferências individuais. O prism justifica-se pela necessidade de um sistema que refine a experiência de descoberta, oferecendo recomendações de filmes personalizadas e eficientes.

A crescente capacidade dos Grandes Modelos de Linguagem (LLMs), como Gemini ou Claude, em processar e interpretar linguagem natural é um fator determinante para a relevância do prism. Ao contrário dos algoritmos de recomendação baseados unicamente em filtragem colaborativa ou de conteúdo, a integração de LLMs permite ao sistema interpretar prompts de contexto e requisições complexas do usuário. Essa capacidade habilita a geração de recomendações mais sofisticadas, contextuais e adaptadas dinamicamente, transformando a interação do usuário com o sistema em um processo mais intuitivo e preciso de descoberta de conteúdo.