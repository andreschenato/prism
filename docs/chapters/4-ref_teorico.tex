\section{Referencial Teórico}
O referencial teórico tem como finalidade apresentar os conceitos e fundamentos que sustentam o desenvolvimento deste projeto. Essa seção busca contextualizar as ferramentas e metodologias utilizadas, evidenciando sua importância para a construção de soluções eficientes, colaborativas e alinhadas às demandas atuais da área de tecnologia. Por meio desse embasamento, garante-se não apenas a fundamentação técnica, mas também a justificativa da escolha dos recursos aplicados ao trabalho.

\subsection{Flutter}
O Flutter é um framework de código aberto desenvolvido pelo Google para a criação de aplicações multiplataforma a partir de uma única base de código. Com ele é possível desenvolver para iOS, Android, Web, Windows, macOS e Linux, garantindo consistência visual e desempenho nativo. Seu diferencial está no uso da linguagem Dart e na arquitetura baseada em widgets, que permitem a construção de interfaces altamente personalizáveis. (FLUTTER, 2025)

Entre suas funcionalidades, destacam-se o Hot Reload, que possibilita visualizar alterações no código em tempo real sem reiniciar a aplicação, acelerando o ciclo de desenvolvimento, além de bibliotecas ricas de componentes e suporte integrado para design responsivo. Essas características tornam o Flutter uma solução moderna para desenvolvimento ágil e multiplataforma. (FLUTTER, 2025)

A escolha do Flutter para este projeto se justifica pela produtividade gerada pela reutilização de código em múltiplas plataformas, pelo alto desempenho proporcionado pelo motor gráfico próprio e pela facilidade de colaboração em equipes que precisam entregar soluções visuais consistentes e escaláveis. Assim, o framework contribui para reduzir custos, agilizar prazos e assegurar qualidade no desenvolvimento. (FLUTTER, 2025)

\subsection{NodeJS}
O Node.js é um ambiente de execução de código JavaScript construído sobre o motor V8 do Google, permitindo que aplicações sejam desenvolvidas e executadas fora do navegador, em servidores ou sistemas back-end (NODE.JS, 2025). Essa capacidade amplia o alcance da linguagem, tornando possível utilizar o mesmo conhecimento de JavaScript tanto no front-end quanto no back-end, promovendo maior produtividade e padronização no desenvolvimento.

Uma das principais características do Node.js é seu modelo de arquitetura orientada a eventos e I/O não bloqueante (non-blocking I/O). Esse modelo permite que o ambiente gerencie múltiplas requisições simultaneamente sem a necessidade de criar novas threads, utilizando o conceito de Event Loop. Dessa forma, aplicações que demandam alto volume de conexões, como APIs, sistemas de chat ou plataformas em tempo real, conseguem operar com baixo consumo de recursos e alta escalabilidade. (NODE.JS, 2025)

Além disso, o Node.js fornece um conjunto robusto de APIs que abrangem manipulação de arquivos, operações de rede, criptografia, execução de processos filhos e controle de streams, permitindo o desenvolvimento de soluções completas e seguras. A modularidade proporcionada pelos módulos do Node também facilita a manutenção do código, a reutilização de componentes e a integração com bibliotecas externas, tornando-o uma escolha estratégica para projetos que exigem flexibilidade e eficiência. (NODE.JS, 2025)

A adoção do Node.js neste projeto justifica-se pela capacidade de unificar o desenvolvimento utilizando uma única linguagem, agilizar o processo de construção do back-end e permitir o desenvolvimento de aplicações escaláveis e de alto desempenho. Sua arquitetura eficiente, combinada com o amplo suporte de APIs e módulos, garante produtividade, facilidade de manutenção e robustez, atendendo às necessidades técnicas e práticas do projeto. (NODE.JS, 2025)

\subsection{Firebase}
O Firebase é uma plataforma de desenvolvimento de aplicativos móveis e web oferecida pelo Google, que fornece uma variedade de ferramentas para facilitar a criação, o gerenciamento e a escalabilidade de aplicativos (FIREBASE, 2025). Entre seus principais serviços, destacam-se o Firebase Authentication, o Cloud Firestore e as Cloud Functions, que, quando integrados, permitem o desenvolvimento de soluções robustas, seguras e eficientes.

\subsection{Firebase Authentication}
Segundo o Firebase (2025), o Firebase Authentication permite autenticar usuários de forma segura e simplificada, oferecendo suporte a métodos de login como e-mail e senha, autenticação por número de telefone e provedores de identidade federada, como Google, Facebook, GitHub e Apple. Além disso, disponibiliza recursos de autenticação anônima e multifatorial, garantindo maior segurança no acesso aos aplicativos. Dessa forma, o serviço simplifica o gerenciamento de credenciais e validação de usuários, proporcionando uma experiência confiável para os usuários. (FIREBASE, 2025)

\subsection{Cloud Firestore}
O Cloud Firestore é um banco de dados NoSQL flexível e escalável, projetado para armazenar e sincronizar dados entre clientes e servidores em tempo real. Ele organiza os dados em documentos e coleções, permitindo consultas complexas, filtragem, ordenação e suporte a transações. O Firestore oferece sincronização automática entre dispositivos, operação offline e atualizações em tempo real, garantindo que os usuários tenham acesso imediato às alterações de dados sem necessidade de atualização manual da aplicação. (FIREBASE, 2025)

\subsection{Cloud Functions}
Segundo o Firebase (2025), as Cloud Functions são funções serverless que permitem executar código em resposta a eventos disparados por outros serviços do Firebase ou de terceiros. No Firestore, essas funções podem ser acionadas por alterações em documentos ou coleções, possibilitando automações como envio de notificações, atualização de registros relacionados ou execução de cálculos em segundo plano. Elas eliminam a necessidade de gerenciar servidores, oferecendo escalabilidade automática e processamento seguro de eventos, sendo fundamentais para o desenvolvimento de funcionalidades avançadas. (FIREBASE, 2025)

A integração entre Authentication, Cloud Firestore e Cloud Functions cria um ecossistema completo para gerenciamento de usuários, armazenamento e processamento de dados de forma segura e eficiente. Essa combinação permite a criação de aplicações escaláveis, seguras e de fácil manutenção, alinhadas às melhores práticas de desenvolvimento moderno. O uso conjunto desses serviços proporciona infraestrutura robusta, facilita a implementação de funcionalidades avançadas e otimiza a gestão de recursos, contribuindo para o sucesso do projeto. (FIREBASE, 2025)

\subsection{APIs Externas}

A integração com APIs externas é um componente essencial para o desenvolvimento de aplicações modernas, permitindo a ampliação das funcionalidades e o acesso a dados e serviços especializados. No contexto deste projeto, a utilização da IMDb API e de uma API de Inteligência Artificial (IA), como Gemini ou Claude, desempenha um papel crucial na oferta de recomendações personalizadas e na melhoria da experiência do usuário.

\subsubsection{IMDb API}
Segundo a IMDb (2025), a API IMDb é uma interface de programação baseada em GraphQL, disponibilizada por meio do AWS Data Exchange, que permite acesso a dados atualizados sobre filmes, séries, atores e outros conteúdos do entretenimento. De acordo com a documentação oficial, a API fornece dados em formato JSON, permitindo consultas precisas sem a necessidade de manipular parâmetros complexos, facilitando a integração com sistemas externos e o desenvolvimento de aplicações interativas. (IMDB, 2025)

De acordo com a IMDb (2025), a API utiliza o GraphQL, uma linguagem de consulta que possibilita requisitar exatamente os dados necessários, evitando sobrecarga de informações desnecessárias. Isso otimiza o desempenho das aplicações, principalmente em cenários de alto volume de consultas. A API também suporta requisições que combinam múltiplos títulos ou nomes em uma única consulta, tornando o acesso às informações mais eficiente e ágil. (IMDB, 2025)

Segundo a documentação oficial, as principais funcionalidades da API incluem:

\begin{itemize}
    \item Consulta de títulos e nomes: permite obter informações detalhadas sobre filmes, séries, episódios e profissionais do entretenimento, incluindo sinopses, classificações, elenco e dados complementares.
    \item Pesquisa avançada: possibilita buscar conteúdos com base em termos específicos, facilitando a localização de informações relevantes.
    \item Dados de bilheteira: disponibiliza informações financeiras sobre filmes, como receitas de bilheteira em diferentes períodos e regiões.
    \item Rankings e métricas: oferece acesso a indicadores como STARmeter e TITLEmeter, que refletem a popularidade de profissionais e títulos.
\end{itemize}

Além disso, a documentação da IMDb destaca que é necessário possuir uma conta na AWS e assinar o produto correspondente no AWS Data Exchange para utilizar a API. Após a assinatura, o desenvolvedor recebe chaves de acesso para realizar requisições, podendo configurar o ambiente e autenticar-se em diferentes linguagens de programação, como Java, Python e TypeScript. (IMDB, 2025)

Segundo a IMDb (2025), a integração da API em projetos permite enriquecer aplicações com dados confiáveis e atualizados do universo do entretenimento, proporcionando aos usuários experiências mais completas e interativas. A utilização da API facilita a manutenção e atualização das informações, uma vez que os dados são obtidos diretamente da fonte oficial em tempo real. (IMDB, 2025)

\subsubsection{APIs de Inteligência Artificial (IA)}

Skiped

\subsection{Git e GitHub}
O GitHub é uma plataforma em nuvem que permite armazenar, compartilhar e colaborar no desenvolvimento de códigos em repositórios. Seu diferencial está no trabalho colaborativo, viabilizado pelo sistema de versionamento Git, que garante o controle das alterações e a integração segura de modificações no código. (GITHUB DOCS, 2025)	

O Git é um sistema de controle de versão que rastreia mudanças em arquivos e facilita o trabalho simultâneo de vários desenvolvedores. Ele possibilita criar ramificações independentes, editar de forma segura e mesclar atualizações à versão principal do projeto, assegurando consistência e organização.

A integração entre Git e GitHub proporciona não apenas o versionamento do código, mas também um ambiente colaborativo para revisão, integração e sincronização de alterações. Essa combinação é essencial em projetos que demandam rastreabilidade, segurança e produtividade, justificando seu uso neste trabalho.

\subsection{Figma}
O Figma é uma ferramenta de design de interface gráfica (UI/UX), colaborativa e baseada na web, que permite criar, compartilhar, prototipar e construir interfaces visuais com outros usuários em tempo real. Seu funcionamento é facilitado por recursos de edição simultânea, armazenamento automático na nuvem e controle de versões integrado, assegurando que todos vejam sempre a versão mais atual do design. (FIGMA, 2025)

Além disso, o Figma reúne diversas funcionalidades em um único ambiente: criação de designs UI (Figma Design), prototipagem interativa (Figma Prototipação), estruturação de sistemas de design (Figma para Design Systems), e ainda outras ferramentas colaterais como FigJam (quadro branco colaborativo), Figma Slides e Figma Make (para gerar protótipos com IA).

A adoção do Figma no projeto justifica-se pela sua capacidade de centralizar processos de design e promover colaboração eficaz. Ele oferece suporte a workflows integrados — desde o desenho inicial até a prototipagem e entrega ao desenvolvedor — com feedback imediato e histórico de alterações. Esse alinhamento entre designers, desenvolvedores e stakeholders acelera a tomada de decisão, aumenta a consistência visual via bibliotecas compartilhadas e reduz retrabalhos, sendo especialmente valioso em projetos que envolvem interfaces interativas e equipes distribuídas. (FIGMA, 2025)
