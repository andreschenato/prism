\section{Referencial Teórico}
O referencial teórico tem como finalidade apresentar os conceitos e fundamentos que sustentam o desenvolvimento deste projeto. Essa seção busca contextualizar as ferramentas e metodologias utilizadas, evidenciando sua importância para a construção de soluções eficientes, colaborativas e alinhadas às demandas atuais da área de tecnologia. Por meio desse embasamento, garante-se não apenas a fundamentação técnica, mas também a justificativa da escolha dos recursos aplicados ao trabalho.

\subsection{Flutter}
O \textit{Flutter} é um \textit{framework} de código aberto desenvolvido pelo \textit{Google} para a criação de aplicações \textit{multiplataforma} a partir de uma única base de código. Com ele é possível desenvolver para \textit{iOS}, \textit{Android}, \textit{Web}, \textit{Windows}, \textit{macOS} e \textit{Linux}, garantindo consistência visual e desempenho nativo. Seu diferencial está no uso da linguagem \textit{Dart} e na arquitetura baseada em \textit{widgets}, que permitem a construção de interfaces altamente personalizáveis (\textit{Figma}, 2025).

Entre suas funcionalidades, destacam-se o \textit{Hot Reload}, que possibilita visualizar alterações no código em tempo real sem reiniciar a aplicação, acelerando o ciclo de desenvolvimento, além de bibliotecas ricas de componentes e suporte integrado para design responsivo. Essas características tornam o \textit{Flutter} uma solução moderna para desenvolvimento ágil e \textit{multiplataforma} (\textit{Flutter}, 2025).

A escolha do \textit{Flutter} para este projeto se justifica pela produtividade gerada pela reutilização de código em múltiplas plataformas, pelo alto desempenho proporcionado pelo motor gráfico próprio e pela facilidade de colaboração em equipes que precisam entregar soluções visuais consistentes e escaláveis. Assim, o \textit{framework} contribui para reduzir custos, agilizar prazos e assegurar qualidade no desenvolvimento (\textit{Flutter}, 2025).

\subsection{NodeJS}
O \textit{Node.js} é um ambiente de execução de código \textit{JavaScript} construído sobre o motor \textit{V8} do \textit{Google}, permitindo que aplicações sejam desenvolvidas e executadas fora do navegador, em servidores ou sistemas \textit{back-end} (\textit{Node.js}, 2025). Essa capacidade amplia o alcance da linguagem, tornando possível utilizar o mesmo conhecimento de \textit{JavaScript} tanto no \textit{front-end} quanto no \textit{back-end}, promovendo maior produtividade e padronização no desenvolvimento.

Uma das principais características do \textit{Node.js} é seu modelo de arquitetura orientada a eventos e \textit{I/O} não bloqueante (\textit{non-blocking I/O}). Esse modelo permite que o ambiente gerencie múltiplas requisições simultaneamente sem a necessidade de criar novas \textit{threads}, utilizando o conceito de \textit{Event Loop}. Dessa forma, aplicações que demandam alto volume de conexões, como \textit{APIs}, sistemas de chat ou plataformas em tempo real, conseguem operar com baixo consumo de recursos e alta escalabilidade (\textit{Node.js}, 2025).

Além disso, o \textit{Node.js} fornece um conjunto robusto de \textit{APIs} que abrangem manipulação de arquivos, operações de rede, criptografia, execução de processos filhos e controle de \textit{streams}, permitindo o desenvolvimento de soluções completas e seguras. A modularidade proporcionada pelos módulos do \textit{Node} também facilita a manutenção do código, a reutilização de componentes e a integração com bibliotecas externas, tornando-o uma escolha estratégica para projetos que exigem flexibilidade e eficiência (\textit{Node.js}, 2025).

A adoção do \textit{Node.js} neste projeto justifica-se pela capacidade de unificar o desenvolvimento utilizando uma única linguagem, agilizar o processo de construção do \textit{back-end} e permitir o desenvolvimento de aplicações escaláveis e de alto desempenho. Sua arquitetura eficiente, combinada com o amplo suporte de \textit{APIs} e módulos, garante produtividade, facilidade de manutenção e robustez, atendendo às necessidades técnicas e práticas do projeto (\textit{Node.js}, 2025).

\subsection{Firebase}
O \textit{Firebase} é uma plataforma de desenvolvimento de aplicativos móveis e \textit{web} oferecida pelo \textit{Google}, que fornece uma variedade de ferramentas para facilitar a criação, o gerenciamento e a escalabilidade de aplicativos (\textit{Google}, 2025). Entre seus principais serviços, destacam-se o \textit{Firebase Authentication}, o \textit{Cloud Firestore} e as \textit{Cloud Functions}, que, quando integrados, permitem o desenvolvimento de soluções robustas, seguras e eficientes.

\subsubsection{Firebase Authentication}
Segundo o \textit{Firebase} (2025), o \textit{Firebase Authentication} permite autenticar usuários de forma segura e simplificada, oferecendo suporte a métodos de login como e-mail e senha, autenticação por número de telefone e provedores de identidade federada, como \textit{Google}, \textit{Facebook}, \textit{GitHub} e \textit{Apple}. Além disso, disponibiliza recursos de autenticação anônima e multifatorial, garantindo maior segurança no acesso aos aplicativos. Dessa forma, o serviço simplifica o gerenciamento de credenciais e validação de usuários, proporcionando uma experiência confiável para os usuários (\textit{Google}, 2025).

\subsubsection{Cloud Firestore}
O \textit{Cloud Firestore} é um banco de dados \textit{NoSQL} flexível e escalável, projetado para armazenar e sincronizar dados entre clientes e servidores em tempo real. Ele organiza os dados em documentos e coleções, permitindo consultas complexas, filtragem, ordenação e suporte a transações. O \textit{Firestore} oferece sincronização automática entre dispositivos, operação \textit{offline} e atualizações em tempo real, garantindo que os usuários tenham acesso imediato às alterações de dados sem necessidade de atualização manual da aplicação (\textit{Google}, 2025).

\subsubsection{Cloud Functions}
Segundo o \textit{Firebase} (2025), as \textit{Cloud Functions} são funções \textit{serverless} que permitem executar código em resposta a eventos disparados por outros serviços do \textit{Firebase} ou de terceiros. No \textit{Firestore}, essas funções podem ser acionadas por alterações em documentos ou coleções, possibilitando automações como envio de notificações, atualização de registros relacionados ou execução de cálculos em segundo plano. Elas eliminam a necessidade de gerenciar servidores, oferecendo escalabilidade automática e processamento seguro de eventos, sendo fundamentais para o desenvolvimento de funcionalidades avançadas.

A integração entre \textit{Authentication}, \textit{Cloud Firestore} e \textit{Cloud Functions} cria um ecossistema completo para gerenciamento de usuários, armazenamento e processamento de dados de forma segura e eficiente. Essa combinação permite a criação de aplicações escaláveis, seguras e de fácil manutenção, alinhadas às melhores práticas de desenvolvimento moderno. O uso conjunto desses serviços proporciona infraestrutura robusta, facilita a implementação de funcionalidades avançadas e otimiza a gestão de recursos, contribuindo para o sucesso do projeto.

\subsection{APIs Externas}
A integração com \textit{APIs} externas é um componente essencial para o desenvolvimento de aplicações modernas, permitindo a ampliação das funcionalidades e o acesso a dados e serviços especializados. No contexto deste projeto, a utilização da \textit{IMDb API} e de uma \textit{API} de Inteligência Artificial (IA), como \textit{Gemini} ou \textit{Claude}, desempenha um papel crucial na oferta de recomendações personalizadas e na melhoria da experiência do usuário.

\subsubsection{IMDb API}
Segundo a \textit{IMDb} (2025), a \textit{API IMDb} é uma interface de programação baseada em \textit{GraphQL}, disponibilizada por meio do \textit{AWS Data Exchange}, que permite acesso a dados atualizados sobre filmes, séries, atores e outros conteúdos do entretenimento. De acordo com a documentação oficial, a \textit{API} fornece dados em formato \textit{JSON}, permitindo consultas precisas sem a necessidade de manipular parâmetros complexos, facilitando a integração com sistemas externos e o desenvolvimento de aplicações interativas (\textit{IMDb}, 2025).

A \textit{IMDb} (2025) também descreve que a \textit{API} utiliza o \textit{GraphQL}, uma linguagem de consulta que possibilita requisitar exatamente os dados necessários, evitando sobrecarga de informações desnecessárias. Isso otimiza o desempenho das aplicações, principalmente em cenários de alto volume de consultas. A \textit{API} também suporta requisições que combinam múltiplos títulos ou nomes em uma única consulta, tornando o acesso às informações mais eficiente e ágil (\textit{IMDb}, 2025).

As principais funcionalidades da \textit{API} incluem:
\begin{itemize}
    \item Consulta de títulos e nomes: permite obter informações detalhadas sobre filmes, séries, episódios e profissionais do entretenimento, incluindo sinopses, classificações, elenco e dados complementares;
    \item Pesquisa avançada: possibilita buscar conteúdos com base em termos específicos, facilitando a localização de informações relevantes;
    \item Dados de bilheteira: disponibiliza informações financeiras sobre filmes, como receitas de bilheteira em diferentes períodos e regiões;
    \item Rankings e métricas: oferece acesso a indicadores como \textit{STARmeter} e \textit{TITLEmeter}, que refletem a popularidade de profissionais e títulos.
\end{itemize}

Além disso, a documentação da \textit{IMDb} destaca que é necessário possuir uma conta na \textit{AWS} e assinar o produto correspondente no \textit{AWS Data Exchange} para utilizar a \textit{API}. Após a assinatura, o desenvolvedor recebe chaves de acesso para realizar requisições, podendo configurar o ambiente e autenticar-se em diferentes linguagens de programação, como \textit{Java}, \textit{Python} e \textit{TypeScript} (\textit{IMDb}, 2025).

Segundo a \textit{IMDb} (2025), a integração da \textit{API} em projetos permite enriquecer aplicações com dados confiáveis e atualizados do universo do entretenimento, proporcionando aos usuários experiências mais completas e interativas. A utilização da \textit{API} facilita a manutenção e atualização das informações, uma vez que os dados são obtidos diretamente da fonte oficial em tempo real (\textit{IMDb}, 2025).

\subsubsection{APIs de Inteligência Artificial (IA)}
A Inteligência Artificial (IA) é um dos pilares centrais do sistema \textit{Prism}, sendo responsável pelo processamento de linguagem natural e pela geração de recomendações personalizadas para cada usuário. O projeto utiliza a \textit{API Google Gemini}, uma plataforma de IA generativa desenvolvida pela \textit{Google DeepMind}, projetada para interpretar texto, compreender contextos e gerar respostas e análises contextualizadas com base em grandes volumes de dados.

O modelo \textit{Gemini}, lançado oficialmente em 2024, representa a evolução dos modelos de linguagem de grande porte (\textit{Large Language Models — LLMs}), capazes de processar texto, código e dados multimodais. Sua arquitetura baseia-se em redes neurais profundas treinadas em um extenso conjunto de dados, o que permite à \textit{API} compreender intenções, sentimentos e preferências expressas pelos usuários, fornecendo recomendações mais precisas e relevantes (\textit{DeepMind}, 2025).

No contexto do \textit{Prism}, a \textit{API Gemini} é integrada ao \textit{backend} por meio das \textit{Cloud Functions} do \textit{Firebase}, utilizando requisições \textit{HTTP} seguras para envio de \textit{prompts} e retorno das respostas processadas. Essa integração possibilita que a aplicação interprete mensagens de texto enviadas pelos usuários, analise suas preferências de filmes e séries e gere sugestões personalizadas com base no histórico de interação e nas avaliações anteriores. Dessa forma, a IA atua como um mecanismo dinâmico de recomendação contextual, indo além dos métodos tradicionais de filtragem colaborativa ou de conteúdo.

Entre as principais vantagens da adoção da \textit{API Gemini}, destacam-se:
\begin{itemize}
    \item \textbf{Compreensão de Linguagem Natural}: a IA é capaz de interpretar descrições subjetivas fornecidas pelos usuários, como “filmes com finais surpreendentes” ou “séries parecidas com \textit{Breaking Bad}”, identificando padrões semânticos e contextuais;
    \item \textbf{Recomendações Personalizadas}: o modelo gera listas de filmes adaptadas às preferências do usuário, combinando dados da \textit{IMDb API} com análises comportamentais;
    \item \textbf{Aprendizado Contínuo}: as respostas da IA podem ser refinadas com base nas interações, aprimorando a acurácia das recomendações ao longo do tempo;
    \item \textbf{Integração Serverless}: a execução é feita de forma totalmente automatizada via \textit{Cloud Functions}, eliminando a necessidade de servidores dedicados e reduzindo custos operacionais.
\end{itemize}

A utilização da \textit{API Gemini} no projeto justifica-se por sua capacidade de combinar o poder dos modelos de linguagem com a praticidade do ecossistema \textit{Firebase}, resultando em um sistema de recomendação inteligente, escalável e de fácil manutenção. Essa integração reforça o caráter inovador do \textit{Prism}, que não apenas exibe dados estáticos, mas interpreta preferências subjetivas, tornando a experiência de descoberta de conteúdo mais fluida, personalizada e contextualizada.

\subsection{Git e GitHub}
O \textit{GitHub} é uma plataforma em nuvem que permite armazenar, compartilhar e colaborar no desenvolvimento de códigos em repositórios. Seu diferencial está no trabalho colaborativo, viabilizado pelo sistema de versionamento \textit{Git}, que garante o controle das alterações e a integração segura de modificações no código (\textit{Docs}, 2025).

O \textit{Git} é um sistema de controle de versão que rastreia mudanças em arquivos e facilita o trabalho simultâneo de vários desenvolvedores. Ele possibilita criar ramificações independentes, editar de forma segura e mesclar atualizações à versão principal do projeto, assegurando consistência e organização.

A integração entre \textit{Git} e \textit{GitHub} proporciona não apenas o versionamento do código, mas também um ambiente colaborativo para revisão, integração e sincronização de alterações. Essa combinação é essencial em projetos que demandam rastreabilidade, segurança e produtividade, justificando seu uso neste trabalho.

\subsection{Figma}
O \textit{Figma} é uma ferramenta de design de interface gráfica (\textit{UI/UX}), colaborativa e baseada na \textit{web}, que permite criar, compartilhar, prototipar e construir interfaces visuais com outros usuários em tempo real. Seu funcionamento é facilitado por recursos de edição simultânea, armazenamento automático na nuvem e controle de versões integrado, assegurando que todos vejam sempre a versão mais atual do design (\textit{Figma}, 2025).

Além disso, o \textit{Figma} reúne diversas funcionalidades em um único ambiente: criação de designs \textit{UI} (\textit{Figma Design}), prototipagem interativa (\textit{Figma Prototipação}), estruturação de sistemas de design (\textit{Figma para Design Systems}), e ainda outras ferramentas colaterais como \textit{FigJam} (quadro branco colaborativo), \textit{Figma Slides} e \textit{Figma Make} (para gerar protótipos com IA).

A adoção do \textit{Figma} no projeto justifica-se pela sua capacidade de centralizar processos de design e promover colaboração eficaz. Ele oferece suporte a \textit{workflows} integrados — desde o desenho inicial até a prototipagem e entrega ao desenvolvedor — com \textit{feedback} imediato e histórico de alterações. Esse alinhamento entre designers, desenvolvedores e \textit{stakeholders} acelera a tomada de decisão, aumenta a consistência visual via bibliotecas compartilhadas e reduz retrabalhos, sendo especialmente valioso em projetos que envolvem interfaces interativas e equipes distribuídas (\textit{Figma}, 2025).
