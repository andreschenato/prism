\section{Referencial Teórico}
O referencial teórico tem como finalidade apresentar os conceitos e fundamentos que sustentam o desenvolvimento deste projeto. Essa seção busca contextualizar as ferramentas e metodologias utilizadas, evidenciando sua importância para a construção de soluções eficientes, colaborativas e alinhadas às demandas atuais da área de tecnologia. Por meio desse embasamento, garante-se não apenas a fundamentação técnica, mas também a justificativa da escolha dos recursos aplicados ao trabalho.

\subsection{Flutter}
O \textit{Flutter} é um \textit{framework} de código aberto desenvolvido pelo \textit{Google} para a criação de aplicações \textit{multiplataforma} a partir de uma única base de código. Com ele é possível desenvolver para \textit{iOS}, \textit{Android}, \textit{Web}, \textit{Windows}, \textit{macOS} e \textit{Linux}, garantindo consistência visual e desempenho nativo. Seu diferencial está no uso da linguagem \textit{Dart} e na arquitetura baseada em \textit{widgets}, que permitem a construção de interfaces altamente personalizáveis (\textit{Flutter}, 2025).

Entre suas funcionalidades, destacam-se o \textit{Hot Reload}, que possibilita visualizar alterações no código em tempo real sem reiniciar a aplicação, acelerando o ciclo de desenvolvimento, além de bibliotecas ricas de componentes e suporte integrado para design responsivo. Essas características tornam o \textit{Flutter} uma solução moderna para desenvolvimento ágil e \textit{multiplataforma} (\textit{Flutter}, 2025).

A escolha do \textit{Flutter} para este projeto se justifica pela produtividade gerada pela reutilização de código em múltiplas plataformas, pelo alto desempenho proporcionado pelo motor gráfico próprio e pela facilidade de colaboração em equipes que precisam entregar soluções visuais consistentes e escaláveis. Assim, o \textit{framework} contribui para reduzir custos, agilizar prazos e assegurar qualidade no desenvolvimento (\textit{Flutter}, 2025).

\subsection{Firebase}
O \textit{Firebase} é uma plataforma de desenvolvimento de aplicativos móveis e \textit{web} oferecida pelo \textit{Google}, que fornece uma variedade de ferramentas para facilitar a criação, o gerenciamento e a escalabilidade de aplicativos (\textit{Google}, 2025). Entre seus principais serviços, destacam-se o \textit{Firebase Authentication}, o \textit{Cloud Firestore} e as \textit{Cloud Functions}, que, quando integrados, permitem o desenvolvimento de soluções robustas, seguras e eficientes.

\subsubsection{Firebase Authentication}
Segundo o \textit{Firebase} (2025), o \textit{Firebase Authentication} permite autenticar usuários de forma segura e simplificada, oferecendo suporte a métodos de login como e-mail e senha, autenticação por número de telefone e provedores de identidade federada, como \textit{Google}, \textit{Facebook}, \textit{GitHub} e \textit{Apple}. Além disso, disponibiliza recursos de autenticação anônima e multifatorial, garantindo maior segurança no acesso aos aplicativos. Dessa forma, o serviço simplifica o gerenciamento de credenciais e validação de usuários, proporcionando uma experiência confiável para os usuários (\textit{Google}, 2025).

\subsubsection{Cloud Firestore}
O \textit{Cloud Firestore} é um banco de dados \textit{NoSQL} flexível e escalável, projetado para armazenar e sincronizar dados entre clientes e servidores em tempo real. Ele organiza os dados em documentos e coleções, permitindo consultas complexas, filtragem, ordenação e suporte a transações. O \textit{Firestore} oferece sincronização automática entre dispositivos, operação \textit{offline} e atualizações em tempo real, garantindo que os usuários tenham acesso imediato às alterações de dados sem necessidade de atualização manual da aplicação (\textit{Google}, 2025).

\subsection{APIs Externas}
A integração com \textit{APIs} externas é um componente essencial no desenvolvimento de aplicações modernas, pois amplia as funcionalidades disponíveis e possibilita o acesso a dados e serviços especializados. No contexto deste projeto, a utilização da \textit{TMDb API} e da \textit{API OpenRouter}, que fornece acesso a modelos de Inteligência Artificial, desempenha um papel fundamental na geração de recomendações personalizadas e na melhoria da experiência do usuário.

\subsubsection{TMDb API}
A \textit{TMDb API} oferece acesso estruturado a um amplo catálogo de metadados sobre filmes, séries, artistas e outros elementos do universo audiovisual. De acordo com sua documentação oficial, a API disponibiliza dados em formato \textit{JSON} por meio de endpoints REST, permitindo consultas diretas a informações como sinopses, avaliações, gêneros, imagens promocionais e tendências globais. Essa abordagem simplificada facilita a integração e reduz a necessidade de manipulação complexa de parâmetros, tornando o consumo das informações mais ágil e eficiente.

A API também permite combinar diferentes tipos de dados em uma única requisição — como detalhes de um título, elenco, trailers e recomendações relacionadas — o que contribui para otimizar o desempenho da aplicação e reduzir a quantidade de chamadas externas. Dessa forma, a \textit{TMDb API} viabiliza a construção de experiências interativas e atualizadas, fundamentais para sistemas de recomendação baseados em preferências do usuário.


As principais funcionalidades da \textit{API} incluem:
\begin{itemize}
    \item Consulta de títulos e nomes: permite obter informações detalhadas sobre filmes, séries, episódios e profissionais do entretenimento, incluindo sinopses, classificações, elenco e dados complementares;
    \item Pesquisa avançada: possibilita buscar conteúdos com base em termos específicos, facilitando a localização de informações relevantes;
    \item Dados de bilheteira: disponibiliza informações financeiras sobre filmes, como receitas de bilheteira em diferentes períodos e regiões;
    \item Rankings e métricas: oferece acesso a indicadores como \textit{STARmeter} e \textit{TITLEmeter}, que refletem a popularidade de profissionais e títulos.
\end{itemize}

Segundo a \textit{TMDb} (2025), a integração da \textit{API} em projetos permite enriquecer aplicações com dados confiáveis e atualizados do universo do entretenimento, proporcionando aos usuários experiências mais completas e interativas. A utilização da \textit{API} facilita a manutenção e atualização das informações, uma vez que os dados são obtidos diretamente da fonte oficial em tempo real (\textit{TMDb}, 2025).

\subsubsection{APIs de Inteligência Artificial (IA)}
A Inteligência Artificial (IA) constitui um dos pilares centrais do sistema \textit{Prism}, sendo responsável pelo processamento de linguagem natural e pela geração de recomendações personalizadas. Para esse propósito, o projeto utiliza a \textit{API OpenRouter}, uma plataforma que funciona como um gateway para múltiplos modelos de IA generativa, incluindo \textit{Gemini}, \textit{Claude}, \textit{Llama} e outros modelos de última geração. O OpenRouter permite selecionar diferentes provedores de modelos, oferecendo flexibilidade, custo-benefício e diversidade de capacidades linguísticas.

Esses modelos pertencem à categoria de \textit{Large Language Models} (LLMs), capazes de interpretar texto, compreender preferências e gerar análises contextualizadas com base em grandes volumes de dados. Sua arquitetura se fundamenta em redes neurais profundas treinadas em conjuntos extensivos de dados multimodais, o que possibilita identificar intenções, estilos narrativos, temas recorrentes e outras nuances relevantes às preferências do usuário.

No \textit{Prism}, a integração com o \textit{OpenRouter} é realizada por meio de requisições \textit{HTTP} seguras enviadas pelo \textit{backend} no próprio cliente \textit{mobile}, que repassa \textit{prompts} estruturados e recebe como retorno recomendações interpretadas pelo modelo escolhido. Essa abordagem permite analisar o histórico de interação do usuário, interpretar descrições subjetivas sobre preferências e produzir sugestões alinhadas ao perfil individual. A IA passa, assim, a atuar como um mecanismo dinâmico de recomendação contextual, complementando os dados fornecidos pela \textit{TMDb API} com uma camada semântica avançada.

Entre as principais vantagens do uso da \textit{API OpenRouter}, destacam-se:
\begin{itemize}
    \item \textbf{Flexibilidade de Modelos}: possibilidade de utilizar diferentes LLMs, escolhendo aquele mais adequado em termos de custo, desempenho ou estilo de resposta;
    \item \textbf{Compreensão de Linguagem Natural}: interpretação de descrições subjetivas, como “filmes com antagonistas complexos” ou “obras com atmosfera semelhante a \textit{Blade Runner}”;
    \item \textbf{Recomendações Personalizadas}: geração de listas de filmes alinhadas ao perfil do usuário, combinando metadados da \textit{TMDb API} com análise semântica das preferências;
    \item \textbf{Integração Simples e \textit{Serverless}}: comunicação direta via \textit{Cloud Functions}, sem necessidade de servidores dedicados;
    \item \textbf{Evolução Contínua}: melhoria na qualidade das respostas à medida que os modelos disponibilizados pelo OpenRouter são atualizados.
\end{itemize}

A adoção do \textit{OpenRouter} no projeto justifica-se por sua capacidade de oferecer múltiplos modelos de IA generativa em uma única interface, garantindo flexibilidade, escalabilidade e facilidade de manutenção. Essa integração amplia o potencial do \textit{Prism}, permitindo que o sistema vá além da exibição de metadados e interprete preferências subjetivas com precisão, tornando a experiência de descoberta de conteúdo mais fluida, personalizada e contextualizada.


\subsection{Git e GitHub}
O \textit{GitHub} é uma plataforma em nuvem que permite armazenar, compartilhar e colaborar no desenvolvimento de códigos em repositórios. Seu diferencial está no trabalho colaborativo, viabilizado pelo sistema de versionamento \textit{Git}, que garante o controle das alterações e a integração segura de modificações no código (\textit{Docs}, 2025).

O \textit{Git} é um sistema de controle de versão que rastreia mudanças em arquivos e facilita o trabalho simultâneo de vários desenvolvedores. Ele possibilita criar ramificações independentes, editar de forma segura e mesclar atualizações à versão principal do projeto, assegurando consistência e organização.

A integração entre \textit{Git} e \textit{GitHub} proporciona não apenas o versionamento do código, mas também um ambiente colaborativo para revisão, integração e sincronização de alterações. Essa combinação é essencial em projetos que demandam rastreabilidade, segurança e produtividade, justificando seu uso neste trabalho.

\subsection{Figma}
O \textit{Figma} é uma ferramenta de design de interface gráfica (\textit{UI/UX}), colaborativa e baseada na \textit{web}, que permite criar, compartilhar, prototipar e construir interfaces visuais com outros usuários em tempo real. Seu funcionamento é facilitado por recursos de edição simultânea, armazenamento automático na nuvem e controle de versões integrado, assegurando que todos vejam sempre a versão mais atual do design (\textit{Figma}, 2025).

Além disso, o \textit{Figma} reúne diversas funcionalidades em um único ambiente: criação de designs \textit{UI} (\textit{Figma Design}), prototipagem interativa (\textit{Figma Prototipação}), estruturação de sistemas de design (\textit{Figma para Design Systems}), e ainda outras ferramentas colaterais como \textit{FigJam} (quadro branco colaborativo), \textit{Figma Slides} e \textit{Figma Make} (para gerar protótipos com IA).

A adoção do \textit{Figma} no projeto justifica-se pela sua capacidade de centralizar processos de design e promover colaboração eficaz. Ele oferece suporte a \textit{workflows} integrados — desde o desenho inicial até a prototipagem e entrega ao desenvolvedor — com \textit{feedback} imediato e histórico de alterações. Esse alinhamento entre designers, desenvolvedores e \textit{stakeholders} acelera a tomada de decisão, aumenta a consistência visual via bibliotecas compartilhadas e reduz retrabalhos, sendo especialmente valioso em projetos que envolvem interfaces interativas e equipes distribuídas (\textit{Figma}, 2025).
