% !TEX root = main.tex

\section{Materiais e Métodos}
O desenvolvimento deste projeto seguiu uma sequência estruturada de etapas, garantindo planejamento adequado e organização durante todo o processo. Inicialmente, foi definido o tema do projeto, estabelecendo o objetivo principal e as necessidades que seriam atendidas pela solução a ser desenvolvida.

Em seguida, realizou-se a seleção das tecnologias mais adequadas para implementação, levando em consideração critérios como escalabilidade, compatibilidade, facilidade de uso e integração entre ferramentas. As tecnologias escolhidas incluíram ferramentas de versionamento e colaboração, design de interfaces, desenvolvimento multiplataforma e gestão de back-end.

Com as tecnologias definidas, iniciou-se a concepção do protótipo de interface no Figma, permitindo estruturar visualmente as telas e funcionalidades do sistema. Essa etapa foi essencial para mapear a experiência do usuário e ajustar fluxos de navegação antes da implementação do código.

Paralelamente, todo o ambiente de desenvolvimento foi configurado e anexado ao GitHub, garantindo controle de versão, colaboração eficiente e registro das alterações realizadas durante o desenvolvimento. Com o ambiente pronto, a equipe passou a programar o sistema, dividindo as tarefas entre os integrantes de acordo com suas especialidades e responsabilidades. Essa divisão permitiu trabalhar em paralelo nas diferentes partes do projeto, assegurando que as funcionalidades fossem desenvolvidas, testadas e integradas de forma organizada e eficiente.

\subsection{Tecnologias e Ferramentas}
O sistema foi desenvolvido com uma stack moderna e coerente com os requisitos de escalabilidade e automação. Destacam-se:

\begin{itemize}
    \item \textbf{Flutter}: Linguagem e framework para desenvolvimento de aplicações mobile multiplataforma, permitindo criar interfaces responsivas e performáticas.
    \item \textbf{Firebase}: Plataforma de backend como serviço (BaaS), utilizada para autenticação de usuários (Firebase Authentication), banco de dados em tempo real e NoSQL (Cloud Firestore) e funções serverless (Cloud Functions) escritas em Node.js.
    \item \textbf{IMDb API}: API externa utilizada para obtenção de metadados sobre filmes e séries, enriquecendo as recomendações com informações detalhadas.
    \item \textbf{APIs de IA (Gemini, Claude)}: Utilizadas para processamento de linguagem natural, permitindo interpretar prompts de contexto e solicitações complexas dos usuários.
    \item \textbf{Figma}: Ferramenta de design colaborativo utilizada para criar protótipos de interface, facilitando a visualização e iteração sobre o design da aplicação.
    \item \textbf{GitHub}: Plataforma de hospedagem de código-fonte e controle de versão, utilizada para gerenciar o desenvolvimento colaborativo do projeto.
\end{itemize}

\subsection{Estratégia de Desenvolvimento}
Adotou-se uma abordagem incremental e modular, com versionamento via GitHub e organização de tarefas no GitHub Projects. ...

\subsection{Padrões e Boas Práticas}
Aplicaram-se práticas como:

\begin{itemize}
    \item \textbf{Repository Pattern}: Para abstração do acesso a dados, facilitando testes e manutenção.
\end{itemize}