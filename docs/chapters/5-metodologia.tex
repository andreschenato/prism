\section{Metodologia}
O desenvolvimento deste projeto seguiu uma sequência estruturada de etapas, garantindo planejamento adequado e organização durante todo o processo. Inicialmente, foi definido o tema do projeto, estabelecendo o objetivo principal e as necessidades que seriam atendidas pela solução a ser desenvolvida.

Em seguida, realizou-se a seleção das tecnologias mais adequadas para implementação, levando em consideração critérios como escalabilidade, compatibilidade, facilidade de uso e integração entre ferramentas. As tecnologias escolhidas incluíram ferramentas de versionamento e colaboração, design de interfaces, desenvolvimento \textit{multiplataforma} e gestão de \textit{back-end}.

Com as tecnologias definidas, iniciou-se a concepção do protótipo de interface no \textit{Figma}, permitindo estruturar visualmente as telas e funcionalidades do sistema. Essa etapa foi essencial para mapear a experiência do usuário e ajustar fluxos de navegação antes da implementação do código.

Paralelamente, todo o ambiente de desenvolvimento foi configurado e anexado ao \textit{GitHub}, garantindo controle de versão, colaboração eficiente e registro das alterações realizadas durante o desenvolvimento. Com o ambiente pronto, a equipe passou a programar o sistema, dividindo as tarefas entre os integrantes de acordo com suas especialidades e responsabilidades. Essa divisão permitiu trabalhar em paralelo nas diferentes partes do projeto, assegurando que as funcionalidades fossem desenvolvidas, testadas e integradas de forma organizada e eficiente.

\subsection{Tecnologias e Ferramentas}
O sistema foi desenvolvido com uma \textit{stack} moderna e coerente com os requisitos de escalabilidade e automação. Destacam-se:

\begin{itemize}
    \item \textbf{\textit{Flutter}}: linguagem e \textit{framework} para desenvolvimento de aplicações \textit{mobile} \textit{multiplataforma}, permitindo criar interfaces responsivas e performáticas;
    \item \textbf{\textit{Firebase}}: plataforma de \textit{backend} como serviço (\textit{BaaS}), utilizada para autenticação de usuários (\textit{Firebase Authentication}) e banco de dados em tempo real \textit{NoSQL} (\textit{Cloud Firestore});
    \item \textbf{\textit{TMDb API}}: \textit{API} externa utilizada para obtenção de metadados sobre filmes e séries, enriquecendo as recomendações com informações detalhadas;
    \item \textbf{\textit{APIs} de IA (\textit{Gemini}, \textit{Claude})}: utilizadas para processamento de linguagem natural, permitindo interpretar \textit{prompts} de contexto e solicitações complexas dos usuários;
    \item \textbf{\textit{Figma}}: ferramenta de design colaborativo utilizada para criar protótipos de interface, facilitando a visualização e iteração sobre o design da aplicação;
    \item \textbf{\textit{GitHub}}: plataforma de hospedagem de código-fonte e controle de versão, utilizada para gerenciar o desenvolvimento colaborativo do projeto.
\end{itemize}
