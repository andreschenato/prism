\section{Desenvolvimento}

A seguir, são apresentados os aspectos técnicos da implementação do sistema, abrangendo sua arquitetura, estrutura interna, funcionalidades desenvolvidas e decisões adotadas ao longo do processo de construção.

\subsection{Arquitetura do Sistema}

O \textit{Prism} adota uma arquitetura modular estruturada em camadas, com divisão clara entre apresentação, processamento e integração externa. Essa abordagem favorece a escalabilidade, o isolamento de responsabilidades e a manutenção contínua do projeto.

A aplicação é organizada da seguinte forma:

\begin{itemize}
    \item \textbf{Camada de Apresentação (\textit{Frontend})} — implementada em \textit{Flutter}, responsável pela interface visual, navegação e interação direta com o usuário. Essa camada se comunica com os serviços de autenticação e banco de dados do \textit{Firebase} para acessar perfis, preferências e listas personalizadas;
    
    \item \textbf{Camada de Dados e Autenticação} — composta pelo \textit{Firebase Authentication} e pelo \textit{Cloud Firestore}. Essa camada gerencia informações como perfis de usuário, preferências, histórico e favoritos, atuando como a base de persistência da aplicação;
    
    \item \textbf{Camada de Processamento (\textit{Backend})} — formada pelo mecanismo interno de recomendação, o \textit{Recommender Engine}, responsável por interpretar os dados armazenados, consolidar preferências e organizar o fluxo de consulta aos serviços externos;
    
    \item \textbf{Camada de Integração} — responsável pela comunicação com a \textit{TMDb API}, utilizada para coleta de metadados de filmes e séries, e com a \textit{OpenRouter API}, que fornece modelos de \textit{IA} capazes de interpretar preferências em linguagem natural e gerar sugestões personalizadas.
\end{itemize}

\begin{figure}[H]
\centering
\includegraphics[width=0.8\textwidth]{images/diagrama-arquiteura.png}
\caption{Arquitetura do Sistema \textit{Prism}}
\label{fig:diagrama-arquitetura}
\end{figure}

Essa arquitetura permite um fluxo contínuo entre cliente, serviços de dados e mecanismos de recomendação. O aplicativo móvel interage com o \textit{Firestore} e com o \textit{Recommender Engine}, enquanto a camada de integração fornece respostas enriquecidas obtidas a partir das \textit{APIs} externas. Dessa forma, o sistema combina o ecossistema \textit{Firebase} com fontes externas especializadas, permitindo recomendações contextualizadas, baixo acoplamento entre módulos e expansão futura sem comprometer o desempenho.

\subsection{Estrutura da Aplicação}

A estrutura interna do projeto segue uma organização modular, típica de aplicações \textit{Flutter} de médio porte, separando de forma explícita as responsabilidades de configuração, navegação, serviços, componentes visuais e funcionalidades. Essa divisão reduz acoplamento, facilita testes e permite evolução incremental dos módulos.

A pasta \texttt{core/} concentra os elementos centrais da aplicação: o cliente HTTP usado para comunicação com as APIs externas (\texttt{api/api\_client.dart}), as configurações de inicialização do \textit{Firebase}, o gerenciador de dependências via \textit{service locator}, o roteamento global e o tema visual. Também abriga o conjunto de \textit{widgets} reutilizáveis, que padronizam a interface e reduzem duplicações.

As funcionalidades de domínio são agrupadas em \texttt{features/}, cada uma contendo suas camadas de \textit{data}, \textit{domain} e \textit{presentation}, seguindo um modelo inspirado em \textit{clean architecture}. Entre os módulos mais relevantes destacam-se:
\begin{itemize}
    \item \textbf{Auth} — responsável por autenticação, integrando o \textit{Firebase Authentication} via fontes de dados e repositórios dedicados;
    \item \textbf{Complete Profile} — gerencia preferências iniciais do usuário, como gêneros, idiomas e país, utilizando fontes do \textit{Firestore};
    \item \textbf{Media List} e \textbf{Details} — realizam consultas à \textit{TMDb API}, tratam modelos de mídia e expõem as informações para as telas de listagem e detalhes;
    \item \textbf{Recommendations} — módulo que interage com a \textit{OpenRouter API} e implementa a lógica do mecanismo de recomendação;
    \item \textbf{Settings} — responsável pelas telas de configurações e informações da conta.
\end{itemize}

Arquivos como \texttt{main.dart}, \texttt{app.dart} e \texttt{firebase\_options.dart} compõem o ponto de entrada da aplicação, inicializando provedores, configurando o ambiente do \textit{Firebase} e montando a estrutura básica do aplicativo.


\subsection{\textit{Frontend} e Experiência do Usuário}
A interface do usuário foi implementada como uma aplicação móvel \textit{multiplataforma}, desenvolvida com o \textit{framework Flutter} e a linguagem \textit{Dart}. Essa escolha permitiu que o sistema fosse executado tanto em dispositivos \textit{Android} quanto \textit{iOS}, mantendo uma experiência consistente.

O design das telas foi prototipado no \textit{Figma}, seguindo boas práticas de \textit{UX/UI} e aplicando o conceito de interface intuitiva e responsiva. O uso do \textit{Hot Reload} do \textit{Flutter} acelerou o desenvolvimento e os testes, permitindo ajustes visuais em tempo real.

As principais telas do sistema \textit{Prism} estão apresentadas a seguir.

\begin{figure}[H]
    \centering
    \begin{minipage}{0.48\textwidth}
        \centering
        \includegraphics[width=\linewidth]{images/login.png}
        \caption*{(a) Tela de Login e Cadastro}
    \end{minipage}%
    \hfill
    \begin{minipage}{0.48\textwidth}
        \centering
        \includegraphics[width=\linewidth]{images/inicial.png}
        \caption*{(b) Tela Inicial}
    \end{minipage}
    
    \label{fig:telas-login-inicial}
\end{figure}

\begin{figure}[H]
    \centering
    \begin{minipage}{0.48\textwidth}
        \centering
        \includegraphics[width=\linewidth]{images/detalhes.png}
        \caption*{(a) Tela de Detalhes do Filme}
    \end{minipage}%
    \hfill
    \begin{minipage}{0.48\textwidth}
        \centering
        \includegraphics[width=\linewidth]{images/favorito.png}
        \caption*{(b) Tela de Favoritos}
    \end{minipage}
   
    \label{fig:telas-detalhes-favoritos}
\end{figure}

\begin{figure}[H]
    \centering
    \includegraphics[width=0.45\textwidth]{images/perfil.png}
    \caption{Tela de Perfil do Usuário.}
    \label{fig:tela-perfil}
\end{figure}

\subsection{Integração com \textit{APIs} Externas}
O \textit{Prism} consome duas fontes externas principais: uma \textbf{\textit{API} de catálogo audiovisual} (para obtenção de metadados de filmes, séries e artistas) e uma \textbf{\textit{API} de Inteligência Artificial}, responsável por interpretar linguagem natural e auxiliar na geração de recomendações personalizadas.

Essas integrações ocorrem diretamente no aplicativo, por meio do módulo \texttt{core/api/api\_client.dart}, que centraliza as requisições \textit{HTTPS}, gerencia autenticação, cabeçalhos e parse das respostas. Os módulos de \textit{features} utilizam fontes específicas (\textit{sources}) para acessar endpoints distintos, mantendo o isolamento entre camadas e facilitando testes e evolução futura.

O serviço de IA é utilizado para analisar descrições inseridas pelo usuário, interpretar preferências declaradas e construir um contexto unificado de recomendação. Já o serviço de catálogo retorna dados estruturados sobre mídias, temporadas, atores e informações relacionadas.

\subsection{Banco de Dados e Persistência}
A persistência é realizada pelo \textbf{\textit{Cloud Firestore}}, que armazena perfis, preferências, listas de itens favoritos e registros produzidos durante o fluxo de recomendações. Cada módulo do aplicativo acessa o banco por meio de suas respectivas fontes de dados dedicadas, como \texttt{profile\_firestore\_source.dart} ou \texttt{favorite\_source.dart}.

O modelo segue a estrutura de coleções típica do Firestore, permitindo organização flexível dos dados e sincronização em tempo real. O suporte nativo a operações \textit{offline} garante continuidade do uso mesmo quando não há conexão estável. Todas as coleções e seus relacionamentos lógicos já estão representados no diagrama geral do sistema.

