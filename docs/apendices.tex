\section{EXEMPLO - Estrutura de arquivos do backend}
\label{apendice:estrutura-backend}

\begin{figure}[H]
\small
\dirtree{%
.1 project-root/.
.2 apps/.
.3 api/.
.4 src/.
.5 app.module.ts.
.5 main.ts.
.3 consumer/.
.4 src/.
.5 app.module.ts.
.5 main.ts.
.2 libs/.
.3 domain/.
.4 src/.
.5 core/.
.6 core.module.ts.
.6 event-emmiter/.
.7 events.module.ts.
.5 events/.
.6 auth.events.ts.
.6 invoice.events.ts.
.6 user.events.ts.
.5 modules/.
.6 auth/.
.6 customers/.
.6 invoices/.
.6 notifications/.
.6 organizations/.
.6 products/.
.6 subscriptions/.
.6 users/.
.5 shared/.
.6 decorators/.
.6 enums/.
.6 middlewares/.
.6 templates/.
.6 utils/.
.3 infra/.
.4 src/.
.5 database/.
.6 database.module.ts.
.6 entities/.
.6 migrations/.
.5 emails/.
.5 rabbitmq/.
.2 docker-compose.dev.yml.
.2 docker-compose.prod.yml.
.2 README.md.
}
\end{figure}
\normalsize