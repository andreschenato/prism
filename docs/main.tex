% !TEX root = main.tex
\documentclass[12pt]{article}
\usepackage{styles/sbc-template}
\usepackage[utf8]{inputenc}

\usepackage{graphicx,url}
\usepackage{dirtree}
\usepackage{float}

\sloppy

\title{Prism\\ Sistema de recomendação de filmes}

\author{
André Vítor Barbosa Schenato -- \textnormal{103586@aluno.uricer.edu.br}\inst{1}
\\Henri A. Grzegozeski -- \textnormal{104114@aluno.uricer.edu.br}\inst{2} 
\\Levi da Rosa Gomes -- \textnormal{103495@aluno.uricer.edu.br}\inst{3} 
\\Ruan Dalla Rosa -- \textnormal{103423@aluno.uricer.edu.br}\inst{4}
}

\address{Universidade Regional Integrada do Alto Uruguai e das Missões – Campus Erechim\\
Erechim -- RS -- Brazil}

\begin{document}

\maketitle

\begin{resumo}
O presente trabalho apresenta o desenvolvimento do \textit{Prism}, uma aplicação mobile voltada à recomendação personalizada de filmes, com o objetivo de otimizar o processo de descoberta de conteúdo audiovisual em plataformas de streaming. A solução foi construída com base em tecnologias modernas e escaláveis, utilizando o framework \textbf{Flutter} para o front-end e o \textbf{Firebase} como infraestrutura de back-end, integrando serviços como \textit{Authentication}, \textit{Cloud Firestore} e \textit{Cloud Functions} em \textbf{Node.js}. Para enriquecer a experiência do usuário, o sistema estabelece conexão com a \textbf{IMDb API} para obtenção de metadados e com uma \textbf{API de Inteligência Artificial}, como o \textbf{Gemini}, responsável por interpretar preferências e gerar recomendações personalizadas por meio de processamento de linguagem natural. O \textit{Prism} busca oferecer uma navegação intuitiva, com telas desenvolvidas no \textbf{Figma}, e uma experiência de uso consistente entre diferentes dispositivos. O resultado é uma aplicação funcional, segura e com potencial de expansão, que demonstra a viabilidade do uso combinado entre IA generativa e tecnologias \textit{serverless} na criação de sistemas de recomendação modernos e eficientes.
\end{resumo}

\noindent\textbf{Palavras-chave:} Recomendação de filmes; Inteligência Artificial; Flutter; Firebase; Node.js.


\section{Introdução}

Este artigo descreve o desenvolvimento do prism, uma aplicação mobile projetada para otimizar o processo de recomendação personalizada de filmes para usuários. O sistema visa mitigar os desafios inerentes à vasta quantidade de conteúdo disponível em plataformas de streaming, proporcionando uma experiência de descoberta de conteúdo audiovisual mais eficiente e alinhada às preferências individuais.

A arquitetura do prism integra tecnologias modernas para garantir escalabilidade, desempenho e uma experiência de usuário rica. A interface da aplicação é construída utilizando o framework Flutter, permitindo o desenvolvimento multi-plataforma e uma interface de usuário responsiva. O backend é implementado sobre a infraestrutura Firebase, compreendendo serviços como Firebase Authentication para gerenciamento de usuários, Cloud Firestore para persistência de dados e Cloud Functions, desenvolvidas em Node.js, para lógica de negócios serverless.

Para enriquecer a capacidade de recomendação, o prism estabelece integração com APIs externas. A IMDb API é utilizada para a coleta de metadados abrangentes sobre filmes e séries, enquanto uma API de Inteligência Artificial (IA), como Gemini ou Claude, é empregada para processamento de linguagem natural. Esta integração permite a interpretação de prompts de contexto e solicitações do usuário, viabilizando a geração de recomendações altamente personalizadas. A interface do usuário é estruturada em seções intuitivas como "Favoritos", "Perfeitos para Você" e "Você Também Pode Gostar", buscando organizar as recomendações em diferentes níveis de relevância percebida. Os protótipos de design da aplicação foram desenvolvidos em Figma, conforme as melhores práticas de User Experience (UX).

\section{Referencial Teórico}
O referencial teórico tem como finalidade apresentar os conceitos e fundamentos que sustentam o desenvolvimento deste projeto. Essa seção busca contextualizar as ferramentas e metodologias utilizadas, evidenciando sua importância para a construção de soluções eficientes, colaborativas e alinhadas às demandas atuais da área de tecnologia. Por meio desse embasamento, garante-se não apenas a fundamentação técnica, mas também a justificativa da escolha dos recursos aplicados ao trabalho.

\subsection{Flutter}
O \textit{Flutter} é um \textit{framework} de código aberto desenvolvido pelo \textit{Google} para a criação de aplicações \textit{multiplataforma} a partir de uma única base de código. Com ele é possível desenvolver para \textit{iOS}, \textit{Android}, \textit{Web}, \textit{Windows}, \textit{macOS} e \textit{Linux}, garantindo consistência visual e desempenho nativo. Seu diferencial está no uso da linguagem \textit{Dart} e na arquitetura baseada em \textit{widgets}, que permitem a construção de interfaces altamente personalizáveis (\textit{Figma}, 2025).

Entre suas funcionalidades, destacam-se o \textit{Hot Reload}, que possibilita visualizar alterações no código em tempo real sem reiniciar a aplicação, acelerando o ciclo de desenvolvimento, além de bibliotecas ricas de componentes e suporte integrado para design responsivo. Essas características tornam o \textit{Flutter} uma solução moderna para desenvolvimento ágil e \textit{multiplataforma} (\textit{Flutter}, 2025).

A escolha do \textit{Flutter} para este projeto se justifica pela produtividade gerada pela reutilização de código em múltiplas plataformas, pelo alto desempenho proporcionado pelo motor gráfico próprio e pela facilidade de colaboração em equipes que precisam entregar soluções visuais consistentes e escaláveis. Assim, o \textit{framework} contribui para reduzir custos, agilizar prazos e assegurar qualidade no desenvolvimento (\textit{Flutter}, 2025).

\subsection{NodeJS}
O \textit{Node.js} é um ambiente de execução de código \textit{JavaScript} construído sobre o motor \textit{V8} do \textit{Google}, permitindo que aplicações sejam desenvolvidas e executadas fora do navegador, em servidores ou sistemas \textit{back-end} (\textit{Node.js}, 2025). Essa capacidade amplia o alcance da linguagem, tornando possível utilizar o mesmo conhecimento de \textit{JavaScript} tanto no \textit{front-end} quanto no \textit{back-end}, promovendo maior produtividade e padronização no desenvolvimento.

Uma das principais características do \textit{Node.js} é seu modelo de arquitetura orientada a eventos e \textit{I/O} não bloqueante (\textit{non-blocking I/O}). Esse modelo permite que o ambiente gerencie múltiplas requisições simultaneamente sem a necessidade de criar novas \textit{threads}, utilizando o conceito de \textit{Event Loop}. Dessa forma, aplicações que demandam alto volume de conexões, como \textit{APIs}, sistemas de chat ou plataformas em tempo real, conseguem operar com baixo consumo de recursos e alta escalabilidade (\textit{Node.js}, 2025).

Além disso, o \textit{Node.js} fornece um conjunto robusto de \textit{APIs} que abrangem manipulação de arquivos, operações de rede, criptografia, execução de processos filhos e controle de \textit{streams}, permitindo o desenvolvimento de soluções completas e seguras. A modularidade proporcionada pelos módulos do \textit{Node} também facilita a manutenção do código, a reutilização de componentes e a integração com bibliotecas externas, tornando-o uma escolha estratégica para projetos que exigem flexibilidade e eficiência (\textit{Node.js}, 2025).

A adoção do \textit{Node.js} neste projeto justifica-se pela capacidade de unificar o desenvolvimento utilizando uma única linguagem, agilizar o processo de construção do \textit{back-end} e permitir o desenvolvimento de aplicações escaláveis e de alto desempenho. Sua arquitetura eficiente, combinada com o amplo suporte de \textit{APIs} e módulos, garante produtividade, facilidade de manutenção e robustez, atendendo às necessidades técnicas e práticas do projeto (\textit{Node.js}, 2025).

\subsection{Firebase}
O \textit{Firebase} é uma plataforma de desenvolvimento de aplicativos móveis e \textit{web} oferecida pelo \textit{Google}, que fornece uma variedade de ferramentas para facilitar a criação, o gerenciamento e a escalabilidade de aplicativos (\textit{Google}, 2025). Entre seus principais serviços, destacam-se o \textit{Firebase Authentication}, o \textit{Cloud Firestore} e as \textit{Cloud Functions}, que, quando integrados, permitem o desenvolvimento de soluções robustas, seguras e eficientes.

\subsubsection{Firebase Authentication}
Segundo o \textit{Firebase} (2025), o \textit{Firebase Authentication} permite autenticar usuários de forma segura e simplificada, oferecendo suporte a métodos de login como e-mail e senha, autenticação por número de telefone e provedores de identidade federada, como \textit{Google}, \textit{Facebook}, \textit{GitHub} e \textit{Apple}. Além disso, disponibiliza recursos de autenticação anônima e multifatorial, garantindo maior segurança no acesso aos aplicativos. Dessa forma, o serviço simplifica o gerenciamento de credenciais e validação de usuários, proporcionando uma experiência confiável para os usuários (\textit{Google}, 2025).

\subsubsection{Cloud Firestore}
O \textit{Cloud Firestore} é um banco de dados \textit{NoSQL} flexível e escalável, projetado para armazenar e sincronizar dados entre clientes e servidores em tempo real. Ele organiza os dados em documentos e coleções, permitindo consultas complexas, filtragem, ordenação e suporte a transações. O \textit{Firestore} oferece sincronização automática entre dispositivos, operação \textit{offline} e atualizações em tempo real, garantindo que os usuários tenham acesso imediato às alterações de dados sem necessidade de atualização manual da aplicação (\textit{Google}, 2025).

\subsubsection{Cloud Functions}
Segundo o \textit{Firebase} (2025), as \textit{Cloud Functions} são funções \textit{serverless} que permitem executar código em resposta a eventos disparados por outros serviços do \textit{Firebase} ou de terceiros. No \textit{Firestore}, essas funções podem ser acionadas por alterações em documentos ou coleções, possibilitando automações como envio de notificações, atualização de registros relacionados ou execução de cálculos em segundo plano. Elas eliminam a necessidade de gerenciar servidores, oferecendo escalabilidade automática e processamento seguro de eventos, sendo fundamentais para o desenvolvimento de funcionalidades avançadas.

A integração entre \textit{Authentication}, \textit{Cloud Firestore} e \textit{Cloud Functions} cria um ecossistema completo para gerenciamento de usuários, armazenamento e processamento de dados de forma segura e eficiente. Essa combinação permite a criação de aplicações escaláveis, seguras e de fácil manutenção, alinhadas às melhores práticas de desenvolvimento moderno. O uso conjunto desses serviços proporciona infraestrutura robusta, facilita a implementação de funcionalidades avançadas e otimiza a gestão de recursos, contribuindo para o sucesso do projeto.

\subsection{APIs Externas}
A integração com \textit{APIs} externas é um componente essencial para o desenvolvimento de aplicações modernas, permitindo a ampliação das funcionalidades e o acesso a dados e serviços especializados. No contexto deste projeto, a utilização da \textit{IMDb API} e de uma \textit{API} de Inteligência Artificial (IA), como \textit{Gemini} ou \textit{Claude}, desempenha um papel crucial na oferta de recomendações personalizadas e na melhoria da experiência do usuário.

\subsubsection{IMDb API}
Segundo a \textit{IMDb} (2025), a \textit{API IMDb} é uma interface de programação baseada em \textit{GraphQL}, disponibilizada por meio do \textit{AWS Data Exchange}, que permite acesso a dados atualizados sobre filmes, séries, atores e outros conteúdos do entretenimento. De acordo com a documentação oficial, a \textit{API} fornece dados em formato \textit{JSON}, permitindo consultas precisas sem a necessidade de manipular parâmetros complexos, facilitando a integração com sistemas externos e o desenvolvimento de aplicações interativas (\textit{IMDb}, 2025).

A \textit{IMDb} (2025) também descreve que a \textit{API} utiliza o \textit{GraphQL}, uma linguagem de consulta que possibilita requisitar exatamente os dados necessários, evitando sobrecarga de informações desnecessárias. Isso otimiza o desempenho das aplicações, principalmente em cenários de alto volume de consultas. A \textit{API} também suporta requisições que combinam múltiplos títulos ou nomes em uma única consulta, tornando o acesso às informações mais eficiente e ágil (\textit{IMDb}, 2025).

As principais funcionalidades da \textit{API} incluem:
\begin{itemize}
    \item Consulta de títulos e nomes: permite obter informações detalhadas sobre filmes, séries, episódios e profissionais do entretenimento, incluindo sinopses, classificações, elenco e dados complementares;
    \item Pesquisa avançada: possibilita buscar conteúdos com base em termos específicos, facilitando a localização de informações relevantes;
    \item Dados de bilheteira: disponibiliza informações financeiras sobre filmes, como receitas de bilheteira em diferentes períodos e regiões;
    \item Rankings e métricas: oferece acesso a indicadores como \textit{STARmeter} e \textit{TITLEmeter}, que refletem a popularidade de profissionais e títulos.
\end{itemize}

Além disso, a documentação da \textit{IMDb} destaca que é necessário possuir uma conta na \textit{AWS} e assinar o produto correspondente no \textit{AWS Data Exchange} para utilizar a \textit{API}. Após a assinatura, o desenvolvedor recebe chaves de acesso para realizar requisições, podendo configurar o ambiente e autenticar-se em diferentes linguagens de programação, como \textit{Java}, \textit{Python} e \textit{TypeScript} (\textit{IMDb}, 2025).

Segundo a \textit{IMDb} (2025), a integração da \textit{API} em projetos permite enriquecer aplicações com dados confiáveis e atualizados do universo do entretenimento, proporcionando aos usuários experiências mais completas e interativas. A utilização da \textit{API} facilita a manutenção e atualização das informações, uma vez que os dados são obtidos diretamente da fonte oficial em tempo real (\textit{IMDb}, 2025).

\subsubsection{APIs de Inteligência Artificial (IA)}
A Inteligência Artificial (IA) é um dos pilares centrais do sistema \textit{Prism}, sendo responsável pelo processamento de linguagem natural e pela geração de recomendações personalizadas para cada usuário. O projeto utiliza a \textit{API Google Gemini}, uma plataforma de IA generativa desenvolvida pela \textit{Google DeepMind}, projetada para interpretar texto, compreender contextos e gerar respostas e análises contextualizadas com base em grandes volumes de dados.

O modelo \textit{Gemini}, lançado oficialmente em 2024, representa a evolução dos modelos de linguagem de grande porte (\textit{Large Language Models — LLMs}), capazes de processar texto, código e dados multimodais. Sua arquitetura baseia-se em redes neurais profundas treinadas em um extenso conjunto de dados, o que permite à \textit{API} compreender intenções, sentimentos e preferências expressas pelos usuários, fornecendo recomendações mais precisas e relevantes (\textit{DeepMind}, 2025).

No contexto do \textit{Prism}, a \textit{API Gemini} é integrada ao \textit{backend} por meio das \textit{Cloud Functions} do \textit{Firebase}, utilizando requisições \textit{HTTP} seguras para envio de \textit{prompts} e retorno das respostas processadas. Essa integração possibilita que a aplicação interprete mensagens de texto enviadas pelos usuários, analise suas preferências de filmes e séries e gere sugestões personalizadas com base no histórico de interação e nas avaliações anteriores. Dessa forma, a IA atua como um mecanismo dinâmico de recomendação contextual, indo além dos métodos tradicionais de filtragem colaborativa ou de conteúdo.

Entre as principais vantagens da adoção da \textit{API Gemini}, destacam-se:
\begin{itemize}
    \item \textbf{Compreensão de Linguagem Natural}: a IA é capaz de interpretar descrições subjetivas fornecidas pelos usuários, como “filmes com finais surpreendentes” ou “séries parecidas com \textit{Breaking Bad}”, identificando padrões semânticos e contextuais;
    \item \textbf{Recomendações Personalizadas}: o modelo gera listas de filmes adaptadas às preferências do usuário, combinando dados da \textit{IMDb API} com análises comportamentais;
    \item \textbf{Aprendizado Contínuo}: as respostas da IA podem ser refinadas com base nas interações, aprimorando a acurácia das recomendações ao longo do tempo;
    \item \textbf{Integração Serverless}: a execução é feita de forma totalmente automatizada via \textit{Cloud Functions}, eliminando a necessidade de servidores dedicados e reduzindo custos operacionais.
\end{itemize}

A utilização da \textit{API Gemini} no projeto justifica-se por sua capacidade de combinar o poder dos modelos de linguagem com a praticidade do ecossistema \textit{Firebase}, resultando em um sistema de recomendação inteligente, escalável e de fácil manutenção. Essa integração reforça o caráter inovador do \textit{Prism}, que não apenas exibe dados estáticos, mas interpreta preferências subjetivas, tornando a experiência de descoberta de conteúdo mais fluida, personalizada e contextualizada.

\subsection{Git e GitHub}
O \textit{GitHub} é uma plataforma em nuvem que permite armazenar, compartilhar e colaborar no desenvolvimento de códigos em repositórios. Seu diferencial está no trabalho colaborativo, viabilizado pelo sistema de versionamento \textit{Git}, que garante o controle das alterações e a integração segura de modificações no código (\textit{Docs}, 2025).

O \textit{Git} é um sistema de controle de versão que rastreia mudanças em arquivos e facilita o trabalho simultâneo de vários desenvolvedores. Ele possibilita criar ramificações independentes, editar de forma segura e mesclar atualizações à versão principal do projeto, assegurando consistência e organização.

A integração entre \textit{Git} e \textit{GitHub} proporciona não apenas o versionamento do código, mas também um ambiente colaborativo para revisão, integração e sincronização de alterações. Essa combinação é essencial em projetos que demandam rastreabilidade, segurança e produtividade, justificando seu uso neste trabalho.

\subsection{Figma}
O \textit{Figma} é uma ferramenta de design de interface gráfica (\textit{UI/UX}), colaborativa e baseada na \textit{web}, que permite criar, compartilhar, prototipar e construir interfaces visuais com outros usuários em tempo real. Seu funcionamento é facilitado por recursos de edição simultânea, armazenamento automático na nuvem e controle de versões integrado, assegurando que todos vejam sempre a versão mais atual do design (\textit{Figma}, 2025).

Além disso, o \textit{Figma} reúne diversas funcionalidades em um único ambiente: criação de designs \textit{UI} (\textit{Figma Design}), prototipagem interativa (\textit{Figma Prototipação}), estruturação de sistemas de design (\textit{Figma para Design Systems}), e ainda outras ferramentas colaterais como \textit{FigJam} (quadro branco colaborativo), \textit{Figma Slides} e \textit{Figma Make} (para gerar protótipos com IA).

A adoção do \textit{Figma} no projeto justifica-se pela sua capacidade de centralizar processos de design e promover colaboração eficaz. Ele oferece suporte a \textit{workflows} integrados — desde o desenho inicial até a prototipagem e entrega ao desenvolvedor — com \textit{feedback} imediato e histórico de alterações. Esse alinhamento entre designers, desenvolvedores e \textit{stakeholders} acelera a tomada de decisão, aumenta a consistência visual via bibliotecas compartilhadas e reduz retrabalhos, sendo especialmente valioso em projetos que envolvem interfaces interativas e equipes distribuídas (\textit{Figma}, 2025).


% !TEX root = main.tex

\section{Materiais e Métodos}
O desenvolvimento deste projeto seguiu uma sequência estruturada de etapas, garantindo planejamento adequado e organização durante todo o processo. Inicialmente, foi definido o tema do projeto, estabelecendo o objetivo principal e as necessidades que seriam atendidas pela solução a ser desenvolvida.

Em seguida, realizou-se a seleção das tecnologias mais adequadas para implementação, levando em consideração critérios como escalabilidade, compatibilidade, facilidade de uso e integração entre ferramentas. As tecnologias escolhidas incluíram ferramentas de versionamento e colaboração, design de interfaces, desenvolvimento multiplataforma e gestão de back-end.

Com as tecnologias definidas, iniciou-se a concepção do protótipo de interface no Figma, permitindo estruturar visualmente as telas e funcionalidades do sistema. Essa etapa foi essencial para mapear a experiência do usuário e ajustar fluxos de navegação antes da implementação do código.

Paralelamente, todo o ambiente de desenvolvimento foi configurado e anexado ao GitHub, garantindo controle de versão, colaboração eficiente e registro das alterações realizadas durante o desenvolvimento. Com o ambiente pronto, a equipe passou a programar o sistema, dividindo as tarefas entre os integrantes de acordo com suas especialidades e responsabilidades. Essa divisão permitiu trabalhar em paralelo nas diferentes partes do projeto, assegurando que as funcionalidades fossem desenvolvidas, testadas e integradas de forma organizada e eficiente.

\subsection{Tecnologias e Ferramentas}
O sistema foi desenvolvido com uma stack moderna e coerente com os requisitos de escalabilidade e automação. Destacam-se:

\begin{itemize}
    \item \textbf{Flutter}: Linguagem e framework para desenvolvimento de aplicações mobile multiplataforma, permitindo criar interfaces responsivas e performáticas.
    \item \textbf{Firebase}: Plataforma de backend como serviço (BaaS), utilizada para autenticação de usuários (Firebase Authentication), banco de dados em tempo real e NoSQL (Cloud Firestore) e funções serverless (Cloud Functions) escritas em Node.js.
    \item \textbf{IMDb API}: API externa utilizada para obtenção de metadados sobre filmes e séries, enriquecendo as recomendações com informações detalhadas.
    \item \textbf{APIs de IA (Gemini, Claude)}: Utilizadas para processamento de linguagem natural, permitindo interpretar prompts de contexto e solicitações complexas dos usuários.
    \item \textbf{Figma}: Ferramenta de design colaborativo utilizada para criar protótipos de interface, facilitando a visualização e iteração sobre o design da aplicação.
    \item \textbf{GitHub}: Plataforma de hospedagem de código-fonte e controle de versão, utilizada para gerenciar o desenvolvimento colaborativo do projeto.
\end{itemize}

\subsection{Estratégia de Desenvolvimento}
Adotou-se uma abordagem incremental e modular, com versionamento via GitHub e organização de tarefas no GitHub Projects. ...

\subsection{Padrões e Boas Práticas}
Aplicaram-se práticas como:

\begin{itemize}
    \item \textbf{Repository Pattern}: Para abstração do acesso a dados, facilitando testes e manutenção.
\end{itemize}

\section{Desenvolvimento}

A seguir, são apresentados os aspectos técnicos da implementação do sistema, abrangendo sua arquitetura, estrutura interna, funcionalidades desenvolvidas e decisões adotadas ao longo do processo de construção.

\subsection{Arquitetura do Sistema}

O \textit{Prism} adota uma arquitetura modular estruturada em camadas, com divisão clara entre apresentação, processamento e integração externa. Essa abordagem favorece a escalabilidade, o isolamento de responsabilidades e a manutenção contínua do projeto.

A aplicação é organizada da seguinte forma:

\begin{itemize}
    \item \textbf{Camada de Apresentação (\textit{Frontend})} — implementada em \textit{Flutter}, responsável pela interface visual, navegação e interação direta com o usuário. Essa camada se comunica com os serviços de autenticação e banco de dados do \textit{Firebase} para acessar perfis, preferências e listas personalizadas;
    
    \item \textbf{Camada de Dados e Autenticação} — composta pelo \textit{Firebase Authentication} e pelo \textit{Cloud Firestore}. Essa camada gerencia informações como perfis de usuário, preferências, histórico e favoritos, atuando como a base de persistência da aplicação;
    
    \item \textbf{Camada de Processamento (\textit{Backend})} — formada pelo mecanismo interno de recomendação, o \textit{Recommender Engine}, responsável por interpretar os dados armazenados, consolidar preferências e organizar o fluxo de consulta aos serviços externos;
    
    \item \textbf{Camada de Integração} — responsável pela comunicação com a \textit{TMDb API}, utilizada para coleta de metadados de filmes e séries, e com a \textit{OpenRouter API}, que fornece modelos de \textit{IA} capazes de interpretar preferências em linguagem natural e gerar sugestões personalizadas.
\end{itemize}

\begin{figure}[H]
\centering
\includegraphics[width=0.8\textwidth]{images/diagrama-arquiteura.png}
\caption{Arquitetura do Sistema \textit{Prism}}
\label{fig:diagrama-arquitetura}
\end{figure}

Essa arquitetura permite um fluxo contínuo entre cliente, serviços de dados e mecanismos de recomendação. O aplicativo móvel interage com o \textit{Firestore} e com o \textit{Recommender Engine}, enquanto a camada de integração fornece respostas enriquecidas obtidas a partir das \textit{APIs} externas. Dessa forma, o sistema combina o ecossistema \textit{Firebase} com fontes externas especializadas, permitindo recomendações contextualizadas, baixo acoplamento entre módulos e expansão futura sem comprometer o desempenho.

\subsection{Estrutura da Aplicação}

A estrutura interna do projeto segue uma organização modular, típica de aplicações \textit{Flutter} de médio porte, separando de forma explícita as responsabilidades de configuração, navegação, serviços, componentes visuais e funcionalidades. Essa divisão reduz acoplamento, facilita testes e permite evolução incremental dos módulos.

A pasta \texttt{core/} concentra os elementos centrais da aplicação: o cliente HTTP usado para comunicação com as APIs externas (\texttt{api/api\_client.dart}), as configurações de inicialização do \textit{Firebase}, o gerenciador de dependências via \textit{service locator}, o roteamento global e o tema visual. Também abriga o conjunto de \textit{widgets} reutilizáveis, que padronizam a interface e reduzem duplicações.

As funcionalidades de domínio são agrupadas em \texttt{features/}, cada uma contendo suas camadas de \textit{data}, \textit{domain} e \textit{presentation}, seguindo um modelo inspirado em \textit{clean architecture}. Entre os módulos mais relevantes destacam-se:
\begin{itemize}
    \item \textbf{Auth} — responsável por autenticação, integrando o \textit{Firebase Authentication} via fontes de dados e repositórios dedicados;
    \item \textbf{Complete Profile} — gerencia preferências iniciais do usuário, como gêneros, idiomas e país, utilizando fontes do \textit{Firestore};
    \item \textbf{Media List} e \textbf{Details} — realizam consultas à \textit{TMDb API}, tratam modelos de mídia e expõem as informações para as telas de listagem e detalhes;
    \item \textbf{Recommendations} — módulo que interage com a \textit{OpenRouter API} e implementa a lógica do mecanismo de recomendação;
    \item \textbf{Settings} — responsável pelas telas de configurações e informações da conta.
\end{itemize}

Arquivos como \texttt{main.dart}, \texttt{app.dart} e \texttt{firebase\_options.dart} compõem o ponto de entrada da aplicação, inicializando provedores, configurando o ambiente do \textit{Firebase} e montando a estrutura básica do aplicativo.


\subsection{\textit{Frontend} e Experiência do Usuário}
A interface do usuário foi implementada como uma aplicação móvel \textit{multiplataforma}, desenvolvida com o \textit{framework Flutter} e a linguagem \textit{Dart}. Essa escolha permitiu que o sistema fosse executado tanto em dispositivos \textit{Android} quanto \textit{iOS}, mantendo uma experiência consistente.

O design das telas foi prototipado no \textit{Figma}, seguindo boas práticas de \textit{UX/UI} e aplicando o conceito de interface intuitiva e responsiva. O uso do \textit{Hot Reload} do \textit{Flutter} acelerou o desenvolvimento e os testes, permitindo ajustes visuais em tempo real.

As principais telas do sistema \textit{Prism} estão apresentadas a seguir.

\begin{figure}[H]
    \centering
    \begin{minipage}{0.48\textwidth}
        \centering
        \includegraphics[width=\linewidth]{images/login.png}
        \caption*{(a) Tela de Login e Cadastro}
    \end{minipage}%
    \hfill
    \begin{minipage}{0.48\textwidth}
        \centering
        \includegraphics[width=\linewidth]{images/inicial.png}
        \caption*{(b) Tela Inicial}
    \end{minipage}
    
    \label{fig:telas-login-inicial}
\end{figure}

\begin{figure}[H]
    \centering
    \begin{minipage}{0.48\textwidth}
        \centering
        \includegraphics[width=\linewidth]{images/detalhes.png}
        \caption*{(a) Tela de Detalhes do Filme}
    \end{minipage}%
    \hfill
    \begin{minipage}{0.48\textwidth}
        \centering
        \includegraphics[width=\linewidth]{images/favorito.png}
        \caption*{(b) Tela de Favoritos}
    \end{minipage}
   
    \label{fig:telas-detalhes-favoritos}
\end{figure}

\begin{figure}[H]
    \centering
    \includegraphics[width=0.45\textwidth]{images/perfil.png}
    \caption{Tela de Perfil do Usuário.}
    \label{fig:tela-perfil}
\end{figure}

\subsection{Integração com \textit{APIs} Externas}
O \textit{Prism} consome duas fontes externas principais: uma \textbf{\textit{API} de catálogo audiovisual} (para obtenção de metadados de filmes, séries e artistas) e uma \textbf{\textit{API} de Inteligência Artificial}, responsável por interpretar linguagem natural e auxiliar na geração de recomendações personalizadas.

Essas integrações ocorrem diretamente no aplicativo, por meio do módulo \texttt{core/api/api\_client.dart}, que centraliza as requisições \textit{HTTPS}, gerencia autenticação, cabeçalhos e parse das respostas. Os módulos de \textit{features} utilizam fontes específicas (\textit{sources}) para acessar endpoints distintos, mantendo o isolamento entre camadas e facilitando testes e evolução futura.

O serviço de IA é utilizado para analisar descrições inseridas pelo usuário, interpretar preferências declaradas e construir um contexto unificado de recomendação. Já o serviço de catálogo retorna dados estruturados sobre mídias, temporadas, atores e informações relacionadas.

\subsection{Banco de Dados e Persistência}
A persistência é realizada pelo \textbf{\textit{Cloud Firestore}}, que armazena perfis, preferências, listas de itens favoritos e registros produzidos durante o fluxo de recomendações. Cada módulo do aplicativo acessa o banco por meio de suas respectivas fontes de dados dedicadas, como \texttt{profile\_firestore\_source.dart} ou \texttt{favorite\_source.dart}.

O modelo segue a estrutura de coleções típica do Firestore, permitindo organização flexível dos dados e sincronização em tempo real. O suporte nativo a operações \textit{offline} garante continuidade do uso mesmo quando não há conexão estável. Todas as coleções e seus relacionamentos lógicos já estão representados no diagrama geral do sistema.



\section{Resultados e Discussões}
Esta seção apresenta os principais resultados obtidos com o desenvolvimento do projeto, bem como os desafios enfrentados durante sua execução. Também são discutidas as contribuições técnicas e as possibilidades de evolução da solução proposta.

\subsection{Resultados}
O desenvolvimento do prism resultou em ...

Aprofundamento.

\subsection{Discussões}
Discussões.

Futuro.

\section{Conclusão}
O projeto ...

\section*{Referências}
\textbf{Figma}. Página inicial do Figma. Disponível em: \url{https://www.figma.com/pt-br/}. Acesso em: 1 set. 2025.\\
\textbf{Flutter}. Documentação oficial do Flutter. Disponível em: \url{https://docs.flutter.dev/}. Acesso em: 1 set. 2025.\\
\textbf{GitHub Docs}. Sobre o GitHub e o Git. Disponível em: \url{https://docs.github.com/pt/get-started/start-your-journey/about-github-and-git}. Acesso em: 20 ago. 2025.\\
\textbf{Google}. Firebase: Ampliar o Cloud Firestore com o Cloud Functions. Disponível em: \url{https://firebase.google.com/docs/firestore/extend-with-functions?hl=pt-br}. Acesso em: 5 set. 2025.\\
\textbf{IMDb}. IMDb API: Visão geral e aplicações. Disponível em: \url{https://developer.imdb.com/documentation/api-documentation/}. Acesso em: 5 set. 2025.\\
\textbf{Node.js Foundation}. Documentação oficial da API Node.js. Disponível em: \url{https://nodejs.org/docs/latest/api/}. Acesso em: 5 set. 2025.\\
\textbf{Google DeepMind}. Gemini API: Documentação oficial. Disponível em: \url{https://ai.google.dev/docs}. Acesso em: 10 out. 2025.





\end{document}
